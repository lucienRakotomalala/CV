\documentclass[11pt,origdate]{lettre}
\usepackage{palatino}
\usepackage[T1]{fontenc}
\usepackage[latin1]{inputenc}
\usepackage[french]{babel}

\begin{document}
	\begin{letter}{Madame Tagada\\ % Destinataire
		           12 rue des fraises\\
		           29300 Uneville} 
		\address{98 Chemin de Lanusse,\\
				31400 Toulouse}
		\name{Lucien RAKOTOMALALA} % Mon nom
		%\location{Mon département dans l'entreprise}
		\telephone{06 33 85 15 32}
		\email{lulu.rakoto@gmail.com}
		\nofax % pas de fax. Alternative: \fax{numero}
		
		\francais % Met les labels en français et le \closing{} en pleine largeur
		          % Variantes: \anglais, \americain, et \allemand
		\pagestyle{empty} % alternatives : plain (numéro de page en pied), headings (entête avec lieu et date)
		
		\conc{\textbf{Candidature de stage}}
		\lieu{Toulouse} % L'endroit d'où j'écris
		\date{\today} % Mis par défaut à la date du jour
		%\signature{Pierre FRITSCH} % Remplacé par défaut par le contenu de \name{}
		%\vref{vos réf.}
		%\nref{nos réf.}
	
		% MISE EN PAGE ET PERSONNALISATION DU PACKAGE
	
		%\marge{15mm} % Marge du côté gauche, default: 15mm
		%\tension{-6} % Espace entre les blocs: -6 resserré, 6 serré
		
		%\renewcommand{\tellabelname}{Téléphone :} 
		%\renewcommand{\ccname}{CC :}
		%\renewcommand{\enclname}{PJ :}
		\renewcommand{\concname}{} % Elimine l'affichage du texte "Objet :"
		\renewcommand{\emaillabelname}{} % Elimine l'affichage du texte "E-mail :"
		
		%\def\openingspace{10mm} % ajuste l'espace vertical autour du champ sujet, default: 1cm
 		\def\sigspace{10mm} % Espacement vertical entre texte et signature(s), default: 1.5cm
		
		\makeatletter
		% BOITE D'ENTETE
		\def\pict@let@width{185}       % default: 185
		\def\pict@let@height{65}       % default: 65
		\def\pict@let@hoffset{0}       % default: 0
		\def\pict@let@voffset{0}       % default: 0
		% TRAIT DE PLIAGE
		\def\rule@hpos{-25}            % default: -25
		\def\rule@vpos{-15}            % default: -15
		\def\rule@length{10}           % default: 10
		% ADRESSE DE L'EXPEDITEUR
		\def\fromaddress@let@hpos{-10} % default: -10
		\def\fromaddress@let@vpos{80}  % default: 70
		                               % je remonte légèrement l'adresse vers le coin haut gauche
		\fromaddress@let@width=69mm    % default: 69
		% LIEU D'EXPEDITION
		\def\fromlieu@let@hpos{90}     % default: 90
		\def\fromlieu@let@vpos{0}      % default: 62 
		                               % je déplace lieu et date sous l'adresse du destinataire
		\fromlieu@let@width=69mm       % default: 69
		% ADRESSE DU DESTINATAIRE
		\def\toaddress@let@hpos{90}    % default: 90
		\def\toaddress@let@vpos{40}    % default: 40
		\toaddress@let@width=80mm      % default: 80
		\makeatother


		% DEBUT DE LA LETTRE
		\opening{Monsieur,}
		Je vous adresse cette lettre pour vous proposer ma candidature à l'offre de stage que vous avez proposé 
		Je suis actuellement en Master 2 Ingénérie des systèmes temps réel et je dois effectué un stage de fin d'études.
		\closing{Veuillez croire, Madame, aux sentiments qu'on a coutume d'écrire au bas des lettres.}
		
		
	\end{letter}
\end{document}
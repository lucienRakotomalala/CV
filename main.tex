% Exemple de CV utilisant la classe moderncv
% Style classic en bleu
% Article complet : http://blog.madrzejewski.com/creer-cv-elegant-latex-moderncv/

\documentclass[11pt,a4paper]{moderncv}
\moderncvtheme[green]{banking}                
\usepackage[utf8]{inputenc}
\usepackage{fontspec,fontawesome}

\usepackage[top=1.1cm, bottom=1.1cm, left=2cm, right=2cm]{geometry}
% Largeur de la colonne pour les dates
\setlength{\hintscolumnwidth}{2.5cm}

\firstname{Lucien}
\familyname{Rakotomalala}

\title{\'Eudiant en Master Automatique à Toulouse}              
\address{98 Chemin de Lanusse}{31200}    
\email{lulu.rakoto@gmail.com}      
\mobile{06 33 85 15 32} 
\extrainfo{22 ans -- \faAutomobile Permis B }
%  \photo[64pt]
\begin{document}
\maketitle

\section{Formation Universitaire et Diplome}
\cventry{2016 - aujourd'hui}{Etudiant en Master Ingénérie des systèmes Temps réel (ISTR)}{Université Paul Sabatier}{Toulouse}{\textit{$1^{er}$ et $2^{nd}$ années}}{}
\cventry{2015 -2016}{License Mention Electronique Electrotechnique et Automatique (EEA)}{Université Paul Sabatier}{Toulouse III}{\textit{}}{}
\cventry{2013 -2015}{DUT Génie \'Electrique et Informatique Industrielle}{IUT 'A' Ponsan}{Toulouse III}{\textit{Spécialité Automatique}}{}
\cventry{2012 - 2013}{Bac général Scientifique}{Lycée Bellevue}{Albi}{\textit{Spécialité Mathématiques}}{}

\section{Compétences en Automatique et systèmes à temps réel}
\cvitem{Systèmes linéaires continus}{Analyse de systèmes linéaires dynamiques représentés par fonction différentielle ou fonction de transfert.\newline Réalisation en modèle d'état pour une commande par retour d'état. \newline 
Applications de méthodes de discrétisation de systèmes ou de commande pour une implémentation sur carte programmable.}
\cvitem{Systèmes à événements discrets}{Modélisation, étude et commande de SED en réseau de PETRI, automates et Machine à états.}
\cvitem{UML}{Conception de systèmes en orienté objet et mise en oeuvre sur différentes applications}

\section{Compétences informatique}
\cvitemwithcomment{MATLAB}{\'Etude de systèmes dynamiques, interface graphique GUI, SIMULINK}{Avancé}
\cvitemwithcomment{Mise en œuvre de commande de SED}{Langage C, VHDL, C++.}{Avancé}
\cvdoubleitem{\'Editeur de texte}{Microsoft Office et \LaTeX}{OS}{\faLinux, \faWindows}
\cvitemwithcomment{Git}{Suivi de projet avec GitHub via l'interface graphique GitKraken}{Débutant}
\section{Langues}
\cvdoubleitem{Anglais}{lu et écrit}{Occitan}{Bilingue pendant l'enfance}

\section{Expérience professionnelle et Stage}
\cventry{2014}{Stage DUT}{NEXTER Electronics}{Mission : maintenance de carte électronique}{\textit{3 mois poursuivie d'un mois en intérim}}{}

\section{Projets Universitaire}
\subsection{Optimisation de paramètres}
\cvlistitem{Projet MASTER 1 TER : Création d'un modèle autour d'un procédé de bacs d'eau pour une optimisation de paramètres}
\section{Centre d'intérêts}
\subsection{Musique}
\cvitemwithcomment{Trompette}{Dans un groupe de +20 musiciens}{Professeur particulier de musique}
\end{document}

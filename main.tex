% Exemple de CV utilisant la classe moderncv
% Style classic en bleu
% Article complet : http://blog.madrzejewski.com/creer-cv-elegant-latex-moderncv/

\documentclass[11pt,a4paper]{moderncv}
\moderncvtheme[green]{banking}                
%\moderncvstyle{classic}
\usepackage[utf8]{inputenc}
\usepackage{fontawesome}

\usepackage[top=1.1cm, bottom=1.1cm, left=2cm, right=2cm]{geometry}
% Largeur de la colonne pour les dates
\setlength{\hintscolumnwidth}{2.5cm}

\firstname{Lucien}
\familyname{Rakotomalala}

%\title{Doctorant ONERA}
\address{38 Rue des Filatiers}{31000}    
\email{lulu.rakoto@gmail.com}      
\mobile{06 33 85 15 32}
\extrainfo{\faBirthdayCake\ 29 Juin 1995}
%  \photo[64pt]

\begin{document}
\maketitle

\section{Formation Universitaire et Diplome}
\cventry{2018 - aujourd'hui}{Doctorant}{ONERA - ISAE}{Toulouse}{Preuve Formelle du calcul réseau}{}
\cventry{2016 - 2018}{Master Ingénérie des systèmes Temps réel (ISTR)}{Université Paul Sabatier}{Toulouse}{}{}
\cventry{2015 -2016}{License Mention Electronique Electrotechnique et Automatique (EEA)}{Université Paul Sabatier}{Toulouse III}{\textit{}}{}
\cventry{2013 -2015}{DUT Génie \'Electrique et Informatique Industrielle}{IUT 'A' Ponsan}{Toulouse III}{\textit{Spécialité Automatique}}{}
\cventry{2012 - 2013}{Bac général Scientifique}{Lycée Bellevue}{Albi}{\textit{Spécialité Mathématiques}}{}

% \section{Formalisation avec un assistant de preuve}
% \cvitem{Coq}{Réécriture d'une base de la théorie du calcul dans un formalisme compris par Coq}

% \section{Compétences en Automatique et systèmes à temps réel}
% \cvitem{Systèmes linéaires continus}{Analyse de systèmes linéaires dynamiques représentés par fonction différentielle ou fonction de transfert.\newline Réalisation en modèle d'état pour une commande par retour d'état. \newline 
% Applications de méthodes de discrétisation de systèmes ou de commande pour une implémentation sur carte programmable.}
% \cvitem{Systèmes à événements discrets}{Modélisation, étude et commande de SED en réseau de PETRI, automates et Machine à états.}

\section{Compétences informatique}
\cvitemwithcomment{MATLAB}{\'Etude de systèmes dynamiques, interface
  graphique GUI, SIMULINK}{Avancé}

\cvitemwithcomment{Coq}{Formalisation de théorie mathématiques,
  automatisation de preuve}{}  

\cvdoubleitem{Langage de programmation}{Java, C, C++}{}{}

\cvdoubleitem{\'Editeur de texte}{Emacs et
  \LaTeX}{OS}{\faLinux, \faWindows}

\section{Langues}
\cvitem{Anglais}{lu et écrit}

\nocite{*}
\bibliographystyle{abbrv}
\bibliography{bibliography}

\section{Expérience professionnelle et Stage}
\cventry{2014}{Stage DUT}{NEXTER Electronics}{Mission : maintenance de carte électronique}{\textit{3 mois poursuivie d'un mois en intérim}}{}
\cventry{2018}{Stage Fin d'étude}{\textsc{Onera} Toulouse}{Sujet : Conception Conjointe de loi de commande et délais réseau}{\textit{6 mois}}{}

\section{Projets Universitaire}
\subsection{Optimisation de paramètres}
\cvlistitem{Projet \textsc{Master} : Création d'un modèle autour d'un procédé de bacs d'eau pour une optimisation de paramètres}
\subsection{Labyrinthe Dynamique}
\cvlistitem{Projet \textsc{Master} : Modélisation d'un labyrinthe dynamique 2D sous forme de graphe et analyse}

\section{Centre d'intérêts}
\subsection{Musique}
\cvitemwithcomment{Trompette}{Dans un groupe de +20 musiciens}{Professeur particulier de musique}
\end{document}
